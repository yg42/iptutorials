\usepackage{tikz}
\usetikzlibrary{shapes,backgrounds,shapes.misc}
% packages de mise en page
\usepackage[newparttoc]{titlesec} % Allows customization of titles
\usepackage{etoolbox}

%%%%%%%%%%%%%%
% cette partie est due à un bug de texlive
\makeatletter
\patchcmd{\ttlh@hang}{\parindent\z@}{\parindent\z@\leavevmode}{}{}
\patchcmd{\ttlh@hang}{\noindent}{}{}{}
\makeatother

%------------------------------------------------------------------------------
%	MAIN TABLE OF CONTENTS
%------------------------------------------------------------------------------

\usepackage{titletoc} % Required for manipulating the table of contents

\contentsmargin{0cm} % Removes the default margin
% 
% \titleformat{\part}[display]{\center\normalfont\huge\bfseries}
%                             {\partname\ \thepart}{20pt}{\Huge} 

\makeatletter
\renewcommand\contentsname{Table of Contents}
\renewcommand\tableofcontents{%
    \if@twocolumn
      \@restonecoltrue\onecolumn
    \else
      \@restonecolfalse
    \fi
    \chapter*{{\color{c_title}\textsc{\contentsname}}% change here the formatting
        \@mkboth{%
           \MakeUppercase\contentsname}{\MakeUppercase\contentsname}}%
    \@starttoc{toc}%
    \if@restonecol\twocolumn\fi
    }
    \makeatother
    
% titlecontents: style for table of contents
% % % Part text styling
% \titlecontents{part}[1.25cm] % left
% {\addvspace{15pt}\huge\sffamily\bfseries} % above-code
% {\color{c_title}\contentslabel[\huge\thecontentslabel]{1.25cm}\color{c_title}} % numbered-entry-format
% {}  % numberless-entry-format
% {\color{c_title}\huge\titlerule*[1pc]{.}\thecontentspage} % filler-page-format
% [] % below-code
\usepackage{framed}
\renewenvironment{leftbar}
  {\def\FrameCommand{\hspace{6em}%
    {\color{c_title}\vrule width 2pt depth 6pt}\hspace{1em}}%
    \MakeFramed{\parshape 1 0cm \dimexpr\textwidth-6em\relax\FrameRestore}\vskip2pt%
  }
 {\endMakeFramed}
 

\titlecontents{part}
  [0em]{\vspace*{2\baselineskip}}
  {\parbox{4.5em}{%
    \hfill\Huge\sffamily\bfseries\color{c_title}\thecontentspage}%
   \vspace*{-2.3\baselineskip}\leftbar\textsc{\color{c_title}\bfseries\partname~\thecontentslabel}\\\sffamily\bfseries\huge}
  {}{\endleftbar}
  
 %% this is used to put the references as a part in the TOC
\makeatletter
\newcommand{\apppart}[1]{%
  \begingroup
  \patchcmd{\@chapter}
   {\addcontentsline{toc}{chapter}}
   {\addcontentsline{toc}{part}}
   {}{}
  \patchcmd{\@chapter}
   {\addcontentsline{toc}{chapter}}
   {\addcontentsline{toc}{part}}
   {}{}
  \part{#1}
  \endgroup
}
\makeatother
%
% chapter
\titlecontents{chapter}[2.5cm] % Indentation
{\large\sffamily\bfseries} % Spacing and font options for chapters
{\color{c_title}\contentslabel[\Large\thecontentslabel]{.75cm}\color{c_title}} % Chapter number
{} 
{\color{c_rule}\normalsize\sffamily\bfseries\;\titlerule*[.5pc]{.}\;\color{c_title}\thecontentspage} % Page number
[]

% Section text styling
\titlecontents{section}[1.25cm] % Indentation
{\addvspace{5pt}\sffamily\bfseries} % Spacing and font options for sections
{\color{c_title}\contentslabel[\thecontentslabel]{1.25cm}} % Section number
{}
{\color{c_rule}\normalsize\sffamily\bfseries\;\titlerule*[.5pc]{.}\;\color{c_page}\thecontentspage} % Page number
[]

% Subsection text styling
\titlecontents{subsection}[1.25cm] % Indentation
{\addvspace{1pt}\sffamily\small} % Spacing and font options for subsections
{\color{c_section}\contentslabel[\thecontentslabel]{1.25cm}} % Subsection number
{}
{\color{c_page}\sffamily\;\titlerule*[.5pc]{.}\;\thecontentspage} % Page number
[] 
% 
% % subsubsection text styling
% \titlecontents{subsubsection}[1.25cm] % Indentation
% {\addvspace{1pt}\sffamily\small} % Spacing and font options for subsections
% {\color{blueemse!60}\contentslabel[\thecontentslabel]{1.5cm}} % Subsection number
% {}
% {\color{blueemse!60}\sffamily\;\titlerule*[.5pc]{.}\;\thecontentspage} % Page number
% [] 
\usepackage{tcolorbox}

\titleformat{\section}%
	{\normalfont\huge\bfseries\color{c_section}}
	{\llap{\tcbox[colback=c_section_colback, colframe=c_section_colframe, coltext=white, on line, boxsep=0pt, left=4pt, right=4pt, top=4pt, bottom=4pt]{\thesection}}}
	{1em}
	{}

% Section text styling
\titleformat{\subsection}% Indentating
{\Large\sffamily\color{c_section}} % Font settings
{\llap{\thesubsection}}
{0.2em} % sep
{} % before

% Subsection text styling
\titleformat{\subsubsection}% Indentation
{\normalfont\sffamily\color{c_section}} % Font settings
{\thesubsubsection}
{1em}
{}
% 
 % Part text styling
\titleformat{\part}% Indentation
{\Huge\bfseries\color{c_title}} % Font settings
{Part \thepart}
{1em}
{}

%----------------------------------------------------------------------------------------
%	PAGE HEADERS
%----------------------------------------------------------------------------------------
\usepackage[papersize={16.64cm,24.64cm},layoutwidth=16cm, layoutheight=24cm, layouthoffset=0.32cm,layoutvoffset=0.32cm, top=1.5cm,bottom=1.5cm,left=1.7cm,right=1.5cm,headsep=1cm]{geometry} % Page margins
%% format Roman
%\usepackage[papersize={6.25in,9.25in},layoutwidth=6in, layoutheight=9in, layouthoffset=0.125in,layoutvoffset=0.125in, top=1.5cm,bottom=1.5cm,left=1.7cm,right=2cm,headsep=1cm]{geometry} % Page margins
%\usepackage[top=2cm,bottom=2cm,left=2.8cm,right=2.8cm,headsep=1cm]{geometry} % Page margins

\usepackage[]{fancyhdr}
\pagestyle{fancyplain}

\renewcommand{\chaptermark}[1]{\markboth{\sffamily\normalsize\bfseries\color{white} #1}{}} % Chapter text font settings
\renewcommand{\sectionmark}[1]{\markright{\sffamily\normalsize\bfseries\thesection\hspace{5pt}\color{white}#1}{}} % Section text font settings
\fancyhf{} 
%\fancyhead[LE,RO]{\sffamily\normalsize\color{c_section}\thepage} % Font setting for the page number in the header
%\fancyhead[LO]{\color{c_section}\rightmark} % Print the nearest section name on the left side of odd pages
%\fancyhead[RE]{\color{c_section}\leftmark} % Print the current chapter name on the right side of even pages
\renewcommand{\headrulewidth}{0pt} % Width of the rule under the header: 0pt means no rule
\addtolength{\headheight}{2.5pt} % Increase the spacing around the header slightly
\renewcommand{\footrulewidth}{0pt} % Removes the rule in the footer
\fancyhead[LE]{%
  \tikz[remember picture,overlay,baseline]
    \fill[c_qbox_colframe](current page.north west|-0,-\dp\strutbox)
    rectangle(current page.north east);%
  \color{white}\leftmark}
  
\fancyhead[LO]{%
  \tikz[remember picture,overlay,baseline]
    \fill[c_qbox_colframe](current page.north west|-0,-\dp\strutbox)
    rectangle(current page.north east);%
  \color{white}\rightmark}
%\fancyhead[C]{\color{white}Middle part}

\fancyhead[R]{\color{white}\thepage}
\fancypagestyle{plain}{\fancyhead{}\renewcommand{\headrulewidth}{0pt}} % Style for when a plain pagestyle is specified
\usepackage{etoolbox}

\makeatletter
\patchcmd{\@fancyhead}{\rlap}{\color{c_section}\rlap}{}{}
\patchcmd{\headrule}{\hrule}{\color{c_rule}\hrule}{}{}
\patchcmd{\@fancyfoot}{\rlap}{\color{c_section}\rlap}{}{}
\patchcmd{\footrule}{\hrule}{\color{c_rule}\hrule}{}{}
\makeatother

% Removes the header from odd empty pages at the end of chapters
\makeatletter
\renewcommand{\cleardoublepage}{
\clearpage\ifodd\c@page\else
\hbox{}
\vspace*{\fill}
\thispagestyle{empty}
\newpage
\fi}


\usepackage{amsmath,amsfonts,amssymb,amsthm} % For including math equations, theorems, symbols, etc

\def\QRCODE{qrcode}
\def\QRPAGE{qrpage}

\def\difficulty{0}
\def\@makechapterhead#1{
\thispagestyle{empty}
{\centering \normalfont\sffamily
\ifnum \c@secnumdepth >\m@ne
\if@mainmatter
\startcontents
\begin{tikzpicture}[remember picture, overlay]
\node at (current page.north west)
{
\begin{tikzpicture}[remember picture,overlay]

\node (dummy)[anchor=north west]  at (-2cm, -2.5cm){}; % pour faire continuer le cadre a l'exterieur
\node (number)[anchor=north west] at (2cm, -2.5cm) {\huge\sffamily\bfseries\textcolor{c_section} {\thechapter}};
\ifthenelse{\difficulty>0}{
\foreach \i in {1,...,\difficulty}{
  \node [star,fill=c_title,anchor=north east,scale=0.5] at (2cm - 0.3*\i cm, -2.5cm) {};
  }
}{TEST}
\node (section) [anchor=north west, text width=\textwidth-1.5cm] at (3.5cm, -2.5cm) {\huge\sffamily\bfseries\textcolor{c_section} {#1}};%\vphantom{plPQq}
 \node (exterieur) [anchor=west,rounded corners=25pt,draw=c_section,draw opacity=1,line width=2pt,inner sep=12pt,fit={(dummy) (number) (section)}] {};

\end{tikzpicture}};
\end{tikzpicture}}\par\vspace*{2cm}
\fi
\fi
}

\newcommand{\genericcorrectionsection}[2]{%
\vspace{1cm}
\noindent\begin{tikzpicture}[remember picture,overlay]
\node (dummy)[anchor=text]  at (-5cm, 0cm){}; % pour faire continuer le cadre a l'exterieur
\node (section) [anchor=text] {\Large\sffamily\bfseries\textcolor{c_section} {\stepcounter{section}\thesection. #1}};
\node (qr) [right=of section,anchor=west] {\href{\QRPAGE}{\includegraphics[height=1.5cm]{\QRCODE}}};
\node (logo) [anchor=text] at (-1.5cm, -0.1cm) {\includegraphics[height=1cm]{#2}};
\begin{scope}[on background layer]\node (exterieur) [anchor=text,rounded corners=10pt,fill=c_section_colback, draw=c_section_colframe,draw opacity=1,line width=2pt,inner sep=4pt,fit={(dummy) (section) (qr)}] {};
\end{scope}

\end{tikzpicture}
\phantom{#1}
\vspace{.6cm}
%\noindent\makebox[\linewidth]{\color{orange}\rule{\textwidth}{2pt}} 
}

\newcommand{\correctionsection}[1]{%
\genericcorrectionsection{#1}{interrogation.pdf}
}
\newcommand{\mcorrectionsection}[1]{%
\genericcorrectionsection{#1}{matlab-logo.png}
}
\newcommand{\pcorrectionsection}[1]{%
\genericcorrectionsection{#1}{python-logo.pdf}
}

