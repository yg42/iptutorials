\section*{About the authors}
\subsection*{Yann GAVET}

received his "Ingénieur Civil des Mines de Saint-Etienne" diploma in 2001. He then obtained a Master of Science and a PhD thesis on the segmentation of human corneal endothelial cells (in 2004 and 2008). He is now an assistant professor at the Saint-Etienne School of Mines, where he teaches computer science and image processing to engineering and master students. He is a member of the PMDM Department of the LGF Laboratory, UMR CNRS 5307, dedicated to granular media analysis and modelisation.

He is particularly interested in the world of free (LIBRE) software in computer science. His research interests include image processing and analysis, stochastic geometry and numerical simulations. He published more than 70 papers in international journals and conference proceedings. He is a member of the Institute of Electrical and Electronics Engineers (IEEE), the International Association for Pattern Recognition (IAPR), International Society for Stereology and Image Analysis (ISSIA). He has worked for CS-SI (Toulouse, France) as an IT engineer, and for Thalès-Angénieux (Saint-Héand, France) as an image processing expert.

\subsection*{Johan DEBAYLE} 

received his M.Sc., Ph.D. and Habilitation degrees in the field of image processing and analysis, in 2002, 2005 and 2012 respectively. Currently, he is a Full Professor at the Ecole Nationale Supérieure des Mines de Saint-Etienne (ENSM-SE) in France, within the SPIN Center and the LGF Laboratory, UMR CNRS 5307, where he leads the PMDM Department interested in image analysis of granular media. In 2015, he was a Visiting Researcher for 3 months at the ITWM Fraunhofer / University of Kaisersleutern in Germany. In 2017 and 2019, he was invited as Guest Lecturer at the University Gadjah Mada, Yogyakarta, Indonesia. He was also Invited Professor at the University of Puebla in Mexico in 2018 and 2019. He is the Head of the Master of Science in Mathematical Imaging and Spatial Pattern Analysis (MISPA) at the ENSM-SE.

His research interests include image processing and analysis, pattern recognition and stochastic geometry. He published more than 120 international papers in international journals and conference proceedings and served as Program committee member in several international conferences (IEEE ICIP, MICCAI, ICIAR…). He has been invited to give a keynote talk in several international conferences (SPIE ICMV, IEEE ISIVC, SPIE-IS\&T EI, SPIE DCS…)
He is Associate Editor for 3 international journals: Pattern Analysis and Applications (Springer), Journal of Electronic Imaging (SPIE) and Image Analysis and Stereology (ISSIA).

He is a member of the International Society for Optics and Photonics (SPIE), International Association for Pattern Recognition (IAPR), International Society for Stereology and Image Analysis (ISSIA) and Senior Member of the Institute of Electrical and Electronics Engineers (IEEE). 

\subsection*{\'ECOLE NATIONALE SUP\'ERIEURE DES MI\-NES DE SAINT-\'ETI\-EN\-NE}

One of the missions of École des Mines de Saint-Étienne, France, is scientific research at the highest level and contributions to companies’ competitiveness. It aims at conveying the economical politics of the country and speeding up the sustainable industrial development by innovation and efficient contributions.

This high level scientific research leads to publications recognized  by the international scientific community. Research and teaching are very closely interwoven and the consequence of this is the attractiveness of our Master’s Degree courses and our renowned Doctoral School.

\begin{center}
 \includegraphics[width=7cm]{emse.pdf}
\end{center}
% 
% \subsection*{Why a CC-By license?}
% The CC-by license is a very unrestrictive license. In particular, it authorizes everyone to reuse the document, to distribute it, even commercially, on the sole condition that the authors are cited. So why would we do that?  
% 
% What we believe is important is to distribute knowledge freely. Our names on the book are more than enough. If the reader can say after studying these pages: "we really do interesting things in Saint-Etienne", well, the objective will be achieved. Our salaries are paid by the French government, we are teacher-researchers, this work belongs to the public. Our hope is that it will serve some people.
