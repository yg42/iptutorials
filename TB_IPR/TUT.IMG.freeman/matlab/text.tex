\def\QRCODE{TB_IPR_TUT.IMG.freeman_matlabqrcode.png}
\def\QRPAGE{http://www.iptutorials.science/tree/master/TB_IPR/TUT.IMG.freeman/matlab}
\mcorrectionsection{Matlab correction}

\subsection{Shape contours}
To generate a simple object, here is an example:
\begin{matlab}
A=zeros(20,20);
A(5:14,10:17)=1;A(2:18,12:16)=1;
\end{matlab}


The contours are computed in 4- or 8-connectivity, see Fig.\ref{fig:freeman:matlab:sampleimage}.
\begin{matlab}
% compute the contours
contours8 = bwperim(A, 4);
contours4 = bwperim(A, 8);
\end{matlab}

\begin{figure}[htbp]
 \centering
 
 \subfloat[Sample object.]{\includegraphics[width=.3\linewidth]{sampleimage.png}}\hfill
 \subfloat[Contour in 4-connectivity.]{\includegraphics[width=.3\linewidth]{contours4.png}}\hfill
  \subfloat[Contour in 8-connectivity.]{\includegraphics[width=.3\linewidth]{contours8.png}}
 \caption{Simple object and its contours in 4- or 8-connectivity.}
 \label{fig:freeman:matlab:sampleimage}
\end{figure}

\subsection{Freeman chain code}
\subsubsection{First point of the shape}
The important thing is to find one point in the contour. The Freeman chain code is sensitive to this choice, but several methods can transform this code so that this first choice does not have any importance.

\begin{matlab}
function [x,y]=firstPoint(A)
% locates first non zero point of the contour A
[r,c]=find(A);
x=r(1); y=c(1);
\end{matlab}

\begin{mwindow}
>>[r0,c0]=firstPoint(A)

r0 =
     5
c0 =
    10
\end{mwindow}

\subsubsection{Freeman chain code}
\begin{matlab}
function code=freeman(A,r0,c0,conn)
% freeman code of a contour.
% A : contour
% r0, c0 : coordinates of 1st point
% conn: connectivity

B=A;
stop=0; % stop condition
point0=[r0,c0];
if (conn==8)
    lut=[1 2 3; 8 0 4; 7 6 5];
else
    lut=[0 2 0; 8 0 4; 0 6 0];
end

% be careful that these LUTs consider coordinates 
% from left to right, top to bottom
%  0
% 0+------- y
%  |
%  |
%  |
% x|
%
lutx=[-1 -1 -1 0 1 1 1 0];
luty=[-1 0 1 1 1 0 -1 -1];
lutcode=[3 2 1 0 7 6 5 4];

nbrepoints=sum(B(:));
code=[];
point=point0;

for indice = 1:nbrepoints
    B(point(1),point(2))=0;
    window=B(point(1)-1:point(1)+1,point(2)-1:point(2)+1);
    window=window.*lut;
    index=max(window(:));
    if (index==0) % no more points ? should link to first point
        B(point0(1),point0(2))=1;
        window=B(point(1)-1:point(1)+1,point(2)-1:point(2)+1);
        window=window.*lut;
        index=max(window(:));
        B(point0(1),point0(2))=0;
    end
    
    % compute coordinates of new point
    point=[point(1)+lutx(index),point(2)+luty(index)];
    
    % add code 
    code(indice) = lutcode(index);
    
end
\end{matlab}

\begin{mwindow}
z4= 6 6 6 6 6 6 6 6 6 0 0 6 6 6 6 0 0 0 0 2 2 2 2 0 2 2 2 2 2 2 2 2 2 4 2 2 2 4 4 4 4 6 6 6 4 4

z8= 6 6 6 6 0 0 0 7 0 0 0 0 0 0 0 0 0 1 0 0 2 2 2 2 4 4 3 2 4 4 4 4 4 4 4 4 4 6 5 4 4 4
\end{mwindow}


\subsection{Normalization}
\subsubsection{Differential code}
\begin{matlab}
function d=codediff(fcc,conn)
% fcc : freeman chain code
% conn: connectivity
sr=circshift(fcc,1);
d=fcc-sr;
d = mod(d, conn);
\end{matlab}

\subsubsection{Normalization}
\begin{matlab}
function code=freeman_normalization(fcc)
% find the lowest number constituted among all the cyclic translations of 
% fcc: freeman code
L = length(fcc);
C = zeros(L);
for i=1:L
    C(i,:)=circshift(fcc, i);
end

% search for minimum number 
% evaluates the value of the number formed by the array, thus, use polyval
% in decimal basis for getting this number
D= zeros(L, 1);
for i=1:L
    D(i) = polyval(C(i,:), 10);
end

[~, ind]=min(D);
code=C(ind(1),:);
\end{matlab}

The differential code is evaluated in \minline{d8}, the normalization gives \minline{shapenumber8}:
\begin{matlab}
[r0,c0]=firstPoint(A);
z8=freeman(contours8,r0,c0,8);
d8=codediff(z8,8);
shapenumber8=freeman_normalization(d8);
\end{matlab}

\begin{mwindow}
d8 = 2 0 0 0 2 0 0 7 1 0 0 0 0 0 0 0 0 1 7 0 2 0 0 0 2 0 7 7 2 0 0 0 0 0 0 0 0 2 7 7 0 0
shapenumber8 = 0 0 0 0 0 0 0 0 1 7 0 2 0 0 0 2 0 7 7 2 0 0 0 0 0 0 0 0 2 7 7 0 0 2 0 0 0 2 0 0 7 1
\end{mwindow}

\subsubsection{Validation}
This validation shows the effect on a different starting point.

\begin{matlab}
% check for another starting point
r0=9;c0=10;
z8=freeman(contours8,r0,c0,8);
d8=codediff(z8,8);
shapenumber8_startchanged=freeman_normalization(d8);
disp('* validation test for starting point changed')
if (shapenumber8-shapenumber8_startchanged)
    disp('   error: different freeman code')
else
    disp('   OK: same freeman code')
end
\end{matlab}

Another test is to verify the result after a rotation. To prevent discretization problems, we use 90 degrees and take the transpose of the matrix.

\begin{matlab}
% check for rotation by 90 deg
contours8rot=contours8';

figure;imshow(contours8rot);

[r0,c0]=firstPoint(contours8rot);
z8=freeman(contours8rot,r0,c0,8);
d8=codediff(z8,8);
shapenumber8_rotated=freeman_normalization(d8);
disp('* validation test after rotation')
if (shapenumber8-shapenumber8_rotated)
    disp('   error: different freeman code')
else
    disp('   OK: same freeman code')
end
\end{matlab}

The same code should be found:
\begin{mwindow}
* validation test for starting point changed
   OK: same freeman code
* validation test after rotation
   OK: same freeman code
\end{mwindow}


\subsection{Geometrical characterization}
\subsubsection{Perimeter for 8-connectivity}
We first need to extract the codes in the diagonal directions and apply a $\sqrt{2}$ factor, then add the number of codes in vertical and horizontal directions.

\begin{matlab}
nb_diag=mod(z8, 2);
nb_diag=sum(nb_diag(:));
nb = length(z8)-nb_diag;
perimeter = nb_diag * sqrt(2) + nb
stats = regionprops(A,'Perimeter')
\end{matlab}

\begin{mwindow}
perimeter =
   43.6569
stats = 
  struct with fields:
    Perimeter: 41.5900
\end{mwindow}


\subsubsection{Area for 8-connectivity}
\begin{matlab}
area=0;
B=0;   
lutB=[0 1 1 1 0 -1 -1 -1];
for i=1:length(z8)
   lutArea=[-B -(B+0.5) 0 (B+0.5) B (B-0.5) 0 -(B-0.5)];
   area=area+lutArea(z8(i)+1);
   B=B+lutB(z8(i)+1);
end
disp(['Area by freeman code: ' num2str(area)]);
disp(['Number of pixels: ' num2str(sum(A(:)))])
\end{matlab}

\begin{mwindow}
Area by freeman code: 93
Number of pixels: 115
\end{mwindow}
