\def\difficulty{2}
\sujet{Morphological skeletonization}
\index{Mathematical Morphology!Skeleton}
\index{Mathematical Morphology!Hit-or-miss Transform}
\index{Mathematical Morphology!Topological Skeleton}

\begin{note}This tutorial aims to skeletonize objects with specific tools from mathematical morphology (thinning, maximum ball\dots).
\end{note}

\noindent The different processes will be applied on the following image:
%\begin{figure}[h]
\begin{center}\includegraphics[width=3.25cm]{mickey.png}
\end{center}
\vspace*{-10pt}
%\end{figure}


\section{Hit-or-miss transform}
The hit-or-miss transformation enables specific pixel configurations to be detected. Based on a pair of disjoint structuring elements $T=(T^1,T^2)$, this transformation is defined as:
\begin{eqnarray}
\eta_T(X)&=&\{x, T_x^1\subseteq X, T_x^2\subseteq X^c\}\\
&=&\epsilon_{T^1}(X)\cap \epsilon_{T^2}(X^c)
\end{eqnarray}
where $\epsilon_{B}(X)$ denotes the erosion of $X$ using the structuring element $B$.

\begin{qbox}
\begin{enumerate}
	\item Implement the hit-or-miss transform.
	\item Test this operator with the following pair of disjoint structuring elements:
	
	\[\begin{array}{|c|c|c|}
	\hline
	+1&+1&+1\\\hline
	0&+1&0\\\hline
	-1&-1&-1\\\hline
	\end{array}
	\quad
	\left(T^1=\begin{array}{|c|c|c|}
	\hline
	1&1&1\\\hline
	0&1&0\\\hline
	0&0&0\\\hline
	\end{array},\quad
	T^2=\begin{array}{|c|c|c|}
	\hline
	0&0&0\\\hline
	0&0&0\\\hline
	1&1&1\\\hline
	\end{array}\,\right)
	\]
	
	where the points with $+1$ (resp. $-1$) belong to $T^1$ (resp. $T^2$). 
\end{enumerate}
\end{qbox}

\section{Thinning and thickening}
Using the hit-or-miss transform, it is possible to make a thinning or thickening of a binary object in the following way:
\begin{eqnarray}
\theta_T(X)&=&X\backslash \eta_T(X)\\
\chi_T(X)&=&X\cup \eta_T(X^c)
\end{eqnarray}
These two operators are dual in the sense that $\theta_T(X)=(\chi_T(X^c))^c$.
\begin{qbox}
\begin{enumerate}
	\item Implement these two transformations.
	\item Test these operators with the previous pair of structuring elements.
\end{enumerate}
\end{qbox}

\section{Topological skeleton}
By using the two following pairs of structuring elements with their rotations (90$^\circ$) in an iterative way (8 configurations are thus defined), the thinning process converges to a resulting object which is homothetic (topologically equivalent) to the initial object.
$$\begin{array}{|c|c|c|}
	\hline
	+1&+1&+1\\\hline
	0&+1&0\\\hline
	-1&-1&-1\\\hline
	\end{array}
	\quad
\begin{array}{|c|c|c|}
	\hline
	0&+1&0\\\hline
	+1&+1&-1\\\hline
	0&-1&-1\\\hline
	\end{array}	
	$$
	
	\begin{qbox}
\begin{enumerate}
	\item Implement this transformation (the convergence has to be satisfied).
	\item Test this operator and comment.
\end{enumerate}
\end{qbox}

\section{Morphological skeleton}
A ball $B_n(x)$ with center $x$ and radius $n$ is maximum with respect to the set $X$ if there exists neither indice $k$ nor centre $y$ such that:
$$
B_n(x)\subseteq B_k(y) \subseteq X
$$
In this way, the morphological skeleton of a set $X$ is constituted by all the centers of maximum balls. Mathematically, it is defined as: 
\begin{eqnarray}
S(X)=\bigcup_{r} \epsilon_{B_r(0)}(X) \backslash \gamma_{B_1(0)}(\epsilon_{B_r(0)}(X))
\end{eqnarray}

\begin{qbox}
\begin{enumerate}
	\item Implement this transformation.
	\item Test this operator and compare it with the topological skeleton.
\end{enumerate}
\end{qbox}
