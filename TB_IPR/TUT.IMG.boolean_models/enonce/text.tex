\def\difficulty{3}
\sujet{Boolean Models}

\index{Spatial Processes!Boolean Models}

\vspace*{-15pt}

\begin{note}This tutorial aims to study a classical model coming from stochastic geometry: the Boolean model. The first objective is to simulate some realizations of a Boolean model of 2-D disks representing a population of overlapped particles. Thereafter, the geometrical characteristics of the individual disks (from a statistical point of view) will be analyzed.  
\end{note}

\vspace*{-10pt}

\begin{figure}[H]
\centering%
\caption{Illustration of 2-D Boolean models observed in a squared window $\Omega$.}%
\subfloat[Boolean model of disks]{
\scalebox{0.75}{\input{tikzBooleanModelDisk}}}\hspace*{1.5cm}
\subfloat[Boolean model of rectangles]{
\scalebox{0.75}{\input{tikzBooleanModelRectangle}}}%
\vspace*{-10pt}%
\label{fig:booleanModel}%
\end{figure}

\vspace*{-18pt}

\section{Simulation of a 2-D Boo\-lean model}
\vspace*{-10pt}
\index{Spatial Processes!Characterization}
Let $\{x_i;i\in \mathbb{N}\}$ be a random collection of points in $\mathbb{R}^2$ forming a stationary Poisson point process with intensity $\gamma>0$. 
Let $Z_0,Z_1,Z_2,\dots$ be independent, identically distributed random 2-D convex bodies (nonempty, compact, convex sets) with distribution $\mathbb{Q}$, which are independent of the point process $\{x_i;i\in \mathbb{N}\}$. The random points $x_1,x_2,\dots$ are the germs and the random sets $Z_1,Z_2,\dots$ are the grains of the Boolean
model. The random set $Z_0$ is called the typical grain. The union of the translated grains:\vspace*{-8pt}
\begin{eqnarray}
Z=\bigcup_{i=1}^{\infty}(Z_i+x_i)
\end{eqnarray}
\vspace*{-12pt}

\noindent is a random closed set, which is called the stationary Boolean model with intensity $\gamma$ and grain distribution $\mathbb{Q}$. The random collection $X=\{Z_1+x_1,Z_2+x_2,\dots\} $ of the shifted grains is the particle process underlying the Boolean model. 

Figure \ref{fig:booleanModel} shows a realization of two different Boolean models $Z$ observed in a compact and convex observation window $\Omega$.

We are going to simulate some realizations of a Boolean model of 2-D disks in a squared observation window. The final simulated model will be represented as a discrete binary image.

\begin{qbox}
\begin{itemize}
\item
Generate a 2-D discrete observation window $\Omega$ of size $500 \times 500$.
\item
Generate the random germs that follows a Poisson law with intensity $\gamma=100/(500*500)$. Take care of the edge effects (a grain with a germ outside the observation window could intersect it!)
\item Generate the random grains (as disks). The disk radius will follow a uniform distribution $\mathcal{U}(10,50)$.
\item Vizualize the realization.
\end{itemize}
\end{qbox}

\begin{mcomment}
\begin{mremark}
 Look at the \matlabregistered{} function \minline{poissrnd} and \minline{randi}.
 
 The \minline{meshgrid} function can be used to generate the disks.
\end{mremark}
\end{mcomment}

\begin{pcomment}
\begin{premark}
 Look at the \pinline{numpy} functions \minline{random.poisson} and \minline{random.randint}.
 
 Use \pinline{skimage.draw.circle} to generate the disks in an array.
\end{premark}
\end{pcomment}
\vspace*{-8pt}
\section{Geometrical characterization of a 2-D Boo\-lean mo\-del}\vspace*{-8pt}

Assume we observe $Z$ in a compact, convex observation window $\Omega$ with positive volume (as shown in Figure \ref{fig:booleanModel}).
Our aim is to extract distributional information from the geometric properties of the sample $Z \cap W$. Thus, we assume we can measure geometric functionals like the perimeter of the sample $Z \cap W$ which is a finite union of convex bodies (a polyconvex set). By a geometric functional $\phi$ we mean a real-valued functional defined on the space of polyconvex sets with specific additional properties. Important examples of geometric functionals are the Minkowski functionals $W_{n}$ that are related in the 2-D space to the measures of area ($A$), perimeter ($P$) and Euler number ($\chi$):
\begin{eqnarray}
W_0&=&A\\
W_1&=&P/2\\
W_2&=&\pi \chi
\end{eqnarray}
%
The boundary of the observation window $\Omega$ has a disturbing effect. It is therefore of advantage to assume a sufficiently large observation
window and to consider only limits as the window tends to infinity. This motivates the introduction of the density (or specific value) of the Boolean model for a geometric functional $\phi$. The density of $\phi$ is the combined spatial and probabilistic mean value:\vspace*{-8pt}
\begin{eqnarray}
\overline{\phi}(Z)=\lim_{r\rightarrow\infty}\frac{\mathbb{E}[\phi(Z\cap r\Omega)]}{W_0(r\Omega)}
\end{eqnarray}
The crucial problem when studying a Boolean model is, that the particles overlap and can therefore not be observed individually.

For this purpose, the Miles formula gives relationships between the global Minkowski densities and the Minkowski densities of the particle.\index{Spatial Processes!Miles Formula}
For isotropic Boolean models and by using the densities of the particle process $X$:
\begin{eqnarray}
\overline{\phi}(X)=\gamma\mathbb{E}[\phi(Z_0)]
\end{eqnarray}
Miles formulas express the observable Minkowski densities $\overline{W}_{n}(Z)$ in terms of the Minkowski densities $\overline{W}_{n}(X)$ and can be inverted.\\
For example in 2-D: \vspace*{-8pt}
\begin{eqnarray}
\overline{W}_0(Z)&=&1-e^{-\overline{W}_0(X)}\\
\overline{W}_1(Z)&=&e^{-\overline{W}_0(X)}\overline{W}_1(X)\label{eq:miles2}\\
\overline{W}_2(Z)&=&e^{-\overline{W}_0(X)}\left(\overline{W}_2(X)-\overline{W}_1(X)^2\right)\label{eq:miles3}
\end{eqnarray}
Practically, in 2-D, $A,P,\chi$ are the measures (area, perimeter, Euler number) obtained in the entire observation, $a,p,x$ are the measures of the typical grain, and $\gamma$ is the density of the process. $\Omega_{size}$ is the area of the observation window $\Omega$.

\begin{align}
\dfrac{A}{\Omega_{size}} & = 1-e^{-\gamma a}\\
\dfrac{P}{\Omega_{size}} & = e^{-\gamma a}\times \gamma p\\
\dfrac{\pi \chi}{\Omega_{size}} & = e^{-\gamma a} \left( \pi\gamma x - \dfrac{1}{4}(\gamma p)^2 \right)
\end{align}
\vspace*{-8pt}
\begin{qbox}
\begin{itemize}
\item Generate different realizations of the previous Boolean model and compute the Minkowski densities of $Z$ (by using the functions done in the tutorial "Integral Geometry").
\item Compute the theoretical Minkowski densities of $Z$ by using the Miles formulas.
\item Compare the computed and theoretical values. 
\end{itemize}
\end{qbox}
