\def\difficulty{1}
\sujet{Topological Description}

\begin{note}The objective of this tutorial is to classify all foreground points of a binary image according to their topological signification: interior, isolated, border\dots. The reader can refer to \cite{Couprie2012}for more details.\end{note}

\noindent The different processes will be realized on the following binary image.
\begin{figure}[H]
\caption{Test}
\centering\includegraphics[width=4cm]{test_topology.jpg}
\end{figure}


\index{Topology!Neighborhood}
\noindent \textbf{Preliminary definitions:}
\begin{itemize}
	\item  $y$ is 4-adjacent to $x$ if $|y_1-x_1|+|y_2-x_2|\leq 1$.
	\item  $y$ is 8-adjacent to $x$ if $\max(|y_1-x_1|,|y_2-x_2|)\leq 1$.
	\item $V_4(x)=\{y:y\text{ is 4-adjacent to } x\}$; $V_4^*(x)=V_4(x)\backslash \{x\}$.
	\item $V_8(x)=\{y:y\text{ is 8-adjacent to } x\}$; $V_8^*(x)=V_8(x)\backslash \{x\}$.

\vspace*{-8pt}
\begin{figure}[H]
\centering\caption{Different neighborhoods. By convention, pixels in white are of value 1, in black of value 0.}%
\subfloat[$V_4(x)$]{
%\begin{minipage}{0.1\textwidth}
\begin{tabular}{|m{0.15cm}|m{0.15cm}|m{0.15cm}|}
\hline
 \cellcolor{black}$\:$ &  $\:$  & \cellcolor{black}$\:$\\ \hline
  & $\!x$ &  \\ \hline
\cellcolor{black}  &    & \cellcolor{black}\\ \hline
\end{tabular}
%\vspace*{0.1cm}
%\end{minipage}
}\hfill
%\hspace{.5cm}
%
\subfloat[$V_8(x)$]{
%\begin{minipage}{0.1\textwidth}
\begin{tabular}[c]{|m{0.15cm}|m{0.15cm}|m{0.15cm}|}
\hline
 \cellcolor{white}$\:$ &  \cellcolor{white}  & \cellcolor{white}$\:$\\ \hline
\cellcolor{white}  & \cellcolor{white} $\!x$ & \cellcolor{white} \\ \hline
  \cellcolor{white}&  \cellcolor{white}  & \cellcolor{white}\\ \hline
\end{tabular}
%\vspace*{0.1cm}
%\end{minipage}
}\hfill
%\hspace{.5cm}
%
\subfloat[$V_4^*(x)$]{
\begin{tabular}{|m{0.15cm}|m{0.15cm}|m{0.15cm}|}
\hline
 \cellcolor{black}$\:$ &  $\:$  & \cellcolor{black}$\:$\\ \hline
  & \cellcolor{black}{\color{white}$\!x$} &  \\ \hline
\cellcolor{black}  &    & \cellcolor{black}\\ \hline
\end{tabular}
}\hfill
%\hspace{.5cm}
%
\subfloat[$V_8^*(x)$]{
%\begin{minipage}{0.1\textwidth}
\begin{tabular}[c]{|m{0.15cm}|m{0.15cm}|m{0.15cm}|}
\hline
 \cellcolor{white}$\:$ &  \cellcolor{white}  & \cellcolor{white}$\:$\\ \hline
\cellcolor{white}  & \cellcolor{black} {\color{white}$\!x$} & \cellcolor{white} \\ \hline
  \cellcolor{white}&  \cellcolor{white}  & \cellcolor{white}\\ \hline
\end{tabular}
%\vspace*{0.1cm}
%\end{minipage}
}%
\vspace*{-8pt}
\end{figure}
\vspace*{-8pt}

\item a n-path is a point sequence $(x_0,\dots,x_k)$ with $x_j$ n-adjacent to $x_{j-1}$ for $j=1,\dots, k$. 
\item two points $x, y\in X$ are n-connected in $X$ if there exists a n-path $(x=x_0,\dots ,x_k=y)$ such that $x_j\in X$. It defines an equivalence relation.
\item the equivalence classes of the previous binary relation are the n-components of $X$.
%\begin{enumerate}
%	\item Cr�er une fonction $Comp_8(X)$ permettant de d�terminer le nombre de composantes 8-connexes (i.e. suivant la topologie $V_8$) de l'ensemble $X$ des points allum�s (objet) d'une image binaire. Utiliser la fonction \matlabregistered{} \texttt{bwlabel}.
%	\item Cr�er une fonction $Comp_4(X)$ permettant de d�terminer le nombre de composantes 4-connexes (i.e. suivant la topologie $V_4$) de l'ensemble $X$ des points allum�s (objet) d'une image binaire. Utiliser la fonction \matlabregistered{} \texttt{bwlabel}.
%\end{enumerate}
%\end{exo}

\end{itemize}

\section{Connectivity numbers}
Let $Comp_n(X)$ be the number of n-components ($n=4$ or $n=8$ within the selected topology $V_4$ or $V_8$) of the set $X$ of foreground points (object). We define the following set:
$$ 
CAdj_n(x,X)=\{C\in Comp_n(X): C \text{ is n-adjacent to } x \}
$$
We select the 8-connectivity for the set $X$ of foreground points (object) and the 4-connectivity for the complementary $\overline{X}$.

\begin{rmq}Warning: the definition of $CAdj_n$ introduces the n-adjacency to the central pixel x. In the case of the following configuration (Fig.\ref{fig:topological_description:enonce:cadj4}), $T_8=2$, $\bar{T}_8=2$ and $TT_8=3$.\end{rmq}
\vspace*{-8pt}
\begin{figure}[H]
\centering\caption{The pixel in the bottom right corner is not C-adjacent-4 to $x$.}%
\subfloat[$X$]{
\begin{tabular}[c]{|m{0.15cm}|m{0.15cm}|m{0.15cm}|}
\hline
 \cellcolor{black}$\:$ &  $\:$  & $\:$\\ \hline
  & \cellcolor{black}$\:$ & \cellcolor{black} \\ \hline
 &  \cellcolor{black}  & \\ \hline
\end{tabular}
}\hspace{15mm}
\subfloat[$V_8^*(x)\cap X$]{
\begin{tabular}[c]{|m{0.15cm}|m{0.15cm}|m{0.15cm}|}
\hline
 $\:$ &  \cellcolor{black}$\:$  & \cellcolor{black}$\:$\\ \hline
 \cellcolor{black} &  &  \\ \hline
 \cellcolor{black} &  & \cellcolor{black}\\ \hline
\end{tabular}
}\vspace*{-8pt}%
\label{fig:topological_description:enonce:cadj4}%
\end{figure}
\vspace*{-8pt}

\begin{qbox}
\begin{enumerate}
	\item Create a function for determining the connectivity number:\\ $T_8(x,X)=\#CAdj_8(x,V_8^* (x)\cap X)$,
	\item Create a function for determining the connectivity number:\\ $\overline{T}_8(x,X)=\#CAdj_4(x,V_8^* (x)\cap \overline{X})$,
	\item Create a function for determining the number: \\$TT_8(x,X)=\#(V_8^*(x)\cap X)$.
\end{enumerate}
Test these functions on some foreground points of the image 'test'.
\end{qbox}

\vspace*{-8pt}

\begin{mcomment}
\begin{mremark}
Use the functions \minline{bwlabel} and \minline{bwlabeln}.
\end{mremark}
\end{mcomment}

\begin{pcomment}
\begin{premark}
Use \pinline{scipy.ndimage.measurements}.
\end{premark}
\end{pcomment}


\section{Topological classification of binary points}
\index{Topology!Classification}
From the connectivity numbers $T_8(x,X)$, $\overline{T}_8(x,X)$ and $TT_8(x,X)$, it is possible to classify a foreground point $x$ within the binary image $X$ according to its topological signification:\vspace*{-8pt}
$$
\begin{array}{|c|c|c|c|}\hline
	T_8(x,X)&\overline{T}_8(x,X)&TT_8(x,X)&\text{Type}\\\hline
	0&1&&\text{isolated point}\\\hline
	1&0&&\text{interior point}\\\hline
	1&1&>1&\text{border point}\\\hline
	1&1&1&\text{end point}\\\hline
	2&2&&\text{2-junction point}\\\hline
	3&3&&\text{3-junction point}\\\hline
	4&4&&\text{4-junction point}\\\hline
\end{array}
$$


The following Fig. \ref{fig:enonce:topological_description:pointsconfiguration} shows the classification of 4 points.
\vspace*{-8pt}
\begin{figure}[H]
\centering\caption{Points configurations.}%
\subfloat[end]{
%\begin{minipage}{0.12\textwidth}
\begin{tabular}{|m{0.15cm}|m{0.15cm}|m{0.15cm}|}
\hline
  \cellcolor{black}$\:$ &  \cellcolor{black}$\:$ & \cellcolor{black}$\:$\\ \hline
\cellcolor{black} & \cellcolor{white} &  \cellcolor{black}\\ \hline
\cellcolor{black}  &  \cellcolor{white}  & \cellcolor{black}\\ \hline
\end{tabular}
%\end{minipage}
}\hfill
%
\subfloat[isolated]{
%\begin{minipage}{0.12\textwidth}
\begin{tabular}[c]{|m{0.15cm}|m{0.15cm}|m{0.15cm}|}
\hline
\cellcolor{black} $\:$ &  \cellcolor{black}$\:$ & \cellcolor{black}$\:$\\ \hline
\cellcolor{black} & \cellcolor{white}  & \cellcolor{black} \\ \hline
 \cellcolor{black} &  \cellcolor{black} & \cellcolor{black}\\ \hline
\end{tabular}
%\vspace*{0.1cm}
%\end{minipage}
}\hfill
%
\subfloat[border]{
%\begin{minipage}{0.12\textwidth}
\begin{tabular}{|m{0.15cm}|m{0.15cm}|m{0.15cm}|}
\hline
 \cellcolor{black}$\:$ &  \cellcolor{black}$\:$  &\cellcolor{black} $\:$\\ \hline
\cellcolor{white}  & \cellcolor{white} & \cellcolor{white} \\ \hline
 \cellcolor{white} &  \cellcolor{white}  & \cellcolor{white}\\ \hline
\end{tabular}
%\vspace*{0.1cm}
%\end{minipage}
}\hfill
%
\subfloat[interior]{
%\begin{minipage}{0.12\textwidth}
\begin{tabular}[c]{|m{0.15cm}|m{0.15cm}|m{0.15cm}|}
\hline
 \cellcolor{white}$\:$ &  \cellcolor{white}$\:$  & \cellcolor{white}$\:$\\ \hline
\cellcolor{white}  & \cellcolor{white}& \cellcolor{white} \\ \hline
  \cellcolor{white}&  \cellcolor{white}  & \cellcolor{white}\\ \hline
\end{tabular}
%\vspace*{0.1cm}
%\end{minipage}
}%
\label{fig:enonce:topological_description:pointsconfiguration}%
\vspace*{-8pt}
\end{figure}
\vspace*{-8pt}
\begin{qbox}
With the help of this table, classify the points of the image 'test'.
\end{qbox}
