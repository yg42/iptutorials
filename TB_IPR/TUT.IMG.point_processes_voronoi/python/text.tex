\def\QRCODE{TB_IPR_TUT.IMG.point_processes_voronoi_pythonqrcode.png}
\def\QRPAGE{http://www.iptutorials.science/tree/master/TB_IPR/TUT.IMG.point_processes_voronoi/python}
\pcorrectionsection{Python correction}


\begin{python}
from scipy.spatial import Voronoi, voronoi_plot_2d, Delaunay, distance
import numpy as np

import matplotlib.pyplot as plt
from shapely import geometry # for polygons, area and perimeter
import networkx as nx # graphs and minimum spanning tree
\end{python}

\subsection{Random tessellations}
Random tessellations are generated following normal standard and uniform distribution. A regular pattern is also employed. They are illustrated in Fig.\ref{fig:point_processes_voronoi:python:patterns}.

\begin{python}
def dist_poisson(N=100):
    points = np.random.rand(N, 2)
    return points

def dist_gaussienne(N=100):
    points = np.random.randn(N, 2)
    return points

def dist_regular(N=100):
    c = np.floor(np.sqrt(N));
    x2, y2 = np.meshgrid(range(int(c)), range(int(c)));
    points = np.vstack([x2.ravel(), y2.ravel()])
    return points.transpose();
\end{python}

\begin{figure}[H]
 \centering\caption{Different point patterns.}%
 \subfloat[Uniform distribution.]{\includegraphics[width=.3\linewidth]{uniform.pdf}}\hfill
 \subfloat[Gaussian distribution.]{\includegraphics[width=.3\linewidth]{gaussian.pdf}}\hfill
 \subfloat[Regular distribution.]{\includegraphics[width=.3\linewidth]{regular.pdf}}%
 \label{fig:point_processes_voronoi:python:patterns}%
\end{figure}

\subsection{Voronoi diagram and analysis}

The Voronoi diagram is simply generated via the following command, with \pinline{points} being generated with the previous functions:
\begin{python}
vor = Voronoi(points);
\end{python}

The two characterization functions RFH and AD are defined on the Voronoi cells.

\begin{python}
def RFH(vor):
    """
    Evaluates Round Factor Homogeneity from voronoi diagram
    """
    rfs=[];
    for cell in vor.regions:
        if cell and -1 not in cell:
            poly = geometry.Polygon([(vor.vertices[p-1]) for p in cell]);
            rfs.append(4*np.pi*poly.area/(poly.length**2));
    res = 1 - np.std(rfs) / np.mean(rfs);            
    return res;
\end{python}

\begin{python}
def AD(vor):
    """
    Evaluates Area Disorder from voronoi diagram
    """
    areas=[];
    for cell in vor.regions:
        if cell and -1 not in cell:
            poly = geometry.Polygon([(vor.vertices[p-1]) for p in cell]);
            areas.append(poly.area);
    res = 1 - 1/(1+np.std(areas) / np.mean(areas));            
    return res;
\end{python}

\subsection{Delaunay triangulation and minimum spanning tree}
The Delaunay triangulation is computed with 
\begin{python}
tri = delaunay(points);
\end{python}

Then, the conversion to networkx graphs is done by the following function. Notice that the \pinline{pdist} function is used to compute the distance between all pairs of vertices, which is definitely not efficient, but probably simplifies the notations.

\begin{python}
def triToNx(tri):
    
    G = nx.Graph()
    d = distance.pdist(tri.points)
    distances = distance.squareform(d)
    for s in tri.simplices:
        G.add_edge(s[0], s[1], weight=distances[s[0], s[1]])
        G.add_edge(s[1], s[2], weight=distances[s[1], s[2]])
        G.add_edge(s[0], s[2], weight=distances[s[0], s[2]])
    return G
\end{python}

Then, the characterization of the triangulation is done by measuring the distances of the edges.
\begin{python}
def characterization(tri):
    """
    Characterization of the Delaunay triangulation (mean and std dev of edges)
    """
    G = triToNx(tri)
    L = [w['weight'] for _, _, w in G.edges(data=True)]
    return np.mean(L), np.std(L)
\end{python}

The characterization of the minimum spanning tree is done with the same kind of function as for the Delaunay graph, and the MST is computed with:
\begin{python}
mst = nx.minimum_spanning_tree(G)
\end{python}

\subsection{Characterization of different realizations}
$n$ realizations of the two distributions (uniform and Gaussian) are simulated. Then, the Voronoi diagram and the Delaunay  triangulation are computed and characterized. The results are presented in Fig.\ref{fig:point_processes_voronoi:python:characterization}. 
% 
% \begin{python}
% def analyse_distributions(n=100):
%     means=[];
%     sigmas=[];
%     mean=[];
%     sigma=[];
%     rfh=[];
%     ad=[];
%     all_points=[]; 
%     colors = plt.cm.get_cmap('Set1', 8);
%     N = 500;
%     for i in range(n):
%         all_points.append(dist_uniform(N));
%         all_points.append(dist_gaussian(N));
%         for idx, points in enumerate(all_points):
%             vor = Voronoi(points);
%             tri = Delaunay(points);
%             ms, ss = mst(tri);
%             means.append(ms);
%             sigmas.append(ss);
%             
%             m, s = characterization(tri);
%             mean.append(m);
%             sigma.append(s);
%             
%             rfh.append(RFH(vor));
%             ad.append(AD(vor));
%             plt.figure(0);
%             plt.plot(means, sigmas, 'o', c=colors(idx));
%             means.clear()
%             sigmas.clear()
%             
%             plt.figure(1);
%             plt.plot(mean, sigma, 'o', c=colors(idx));
%             mean.clear()
%             sigma.clear()
%      
%             plt.figure(2);
%             plt.plot(ad, rfh, 'o', c=colors(idx));
%             rfh.clear()
%             ad.clear()
%         
%         all_points.clear()
% \end{python}

\begin{figure}[H]
 \centering\caption{Characterization of several point processes. Each color represent a different process, and each point represent one realization. These characterizations can enhance a difference between the spatial distributions of the points.}%
 \subfloat[Characterization by the mean and standard deviation of the lengths of the Delaunay triangulation.]{\includegraphics[width=.45\linewidth]{charac_diagrams.pdf}}\hfill
 \subfloat[Characterization of the minimum spanning tree of the Delaunay triangulation by the mean and standard deviation of the lengths of the MST.]{\includegraphics[width=.45\linewidth]{charac_mst.pdf}}
 
 \subfloat[AD and RFH on the Voronoi diagram.]{\includegraphics[width=.45\linewidth]{ad_rfh.pdf}}%
 \label{fig:point_processes_voronoi:python:characterization}%
\end{figure}
