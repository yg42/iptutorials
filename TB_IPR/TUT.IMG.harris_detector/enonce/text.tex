\def\difficulty{1}
\sujet{Harris corner detector}
\index{Features!Harris Corner Detector}
\begin{note}The aim of this tutorial is to develop a simple Harris corner detector. This is the first step in pattern matching, generally followed by a feature descriptor construction, and a matching process.\end{note}

\vspace*{-10pt}

\section{Corner detector and cornerness measure}

\vspace*{-10pt}

\begin{mcomment}
\begin{mremark}
 Use \minline{imgradientxy} and \minline{imgaussfilt} with a scale parameter $\sigma$ that will constrain the size of the window $W$.
\end{mremark}
\end{mcomment}

\begin{pcomment}
\begin{premark}
 Use the \pinline{sobel} and \pinline{gaussian_filter} from the \pinline{scipy.ndimage} module, with a scale parameter $\sigma$.
\end{premark}
\end{pcomment}

\vspace*{-3pt}

\subsection{Gradient evaluation}

The Harris corner detector is based on the gradients of the image, $I_x$ and $I_y$ in x and y directions, respectively.

\begin{qbox}
 Apply a Sobel gradient in both directions in order to compute $I_x$ and $I_y$.
\end{qbox}


\subsection{Structure tensor}
The structure tensor is defined (for each point of coordinates $(u,v)$) by the following matrix. The coefficients $\omega$ follow a gaussian law, and each summation represents a gaussian filtering process. $W$ is an operating window.\vspace*{-7pt}
\[
M(u,v)=
\begin{bmatrix}
 I_x(u,v)^2 & I_x(u,v) I_y(u,v) \\
 I_x(u,v) I_y(u,v) &  I_y(u,v)^2
\end{bmatrix} \]
This defines 4 matrices $M_1, M_2, M_3, M_4$.



\begin{qbox}
 \begin{itemize}
  \item Using the Sobel gradients in horizontal and vertical direction, compute $M_1$ to $M_4$.
  \item Use a gaussian filter (with parameter $\sigma$) in order to smooth the matrices $M_1$ to $M_4$.
 \end{itemize}

\end{qbox}

\vspace*{-2pt}

\subsection{Cornerness measure}
The cornerness measure $C$, as proposed by Harris and Stephens, is defined as follows for every pixel of coordinates $(u,v)$:\vspace*{-8pt}
\[ 
C(u,v) = \textrm{det}(M(u,v))-K \textrm{trace}(M(u,v))^2\vspace*{-8pt}
\]
with $K$ between 0.04 and 0.15.

\begin{qbox}
 By using the capacity of numpy to make operations on all values of arrays, compute $C$ for all pixels and display it for several scales $\sigma$ (parameter used in the gaussian filter).
\end{qbox}

\section{Corners detection}
A so-called Harris corner is the result of keeping only local maxima above a certain threshold value.  You can use the checkerboard image for testing, or load the sweden road sign image Fig.\ref{fig:harrs:enonce:road_sign}.

\begin{mcomment}Use the command \minline{checkerboard(8)>.5} for generating a checkerboard figure, or load the road sign:
	\begin{matlab}
		I = imread('sweden_road.png');
	\end{matlab}
\end{mcomment}

\begin{pcomment}
	
Use the following function to generate a checkerboard pattern.	
\begin{python}
def checkerboard(nb_x=2, nb_y=2, s=10):
	"""
	checkerboard generation
	a grid of size 2*nb_x by 2*nb_y is generated
	each square has s pixels.
	"""
	C = 255*np.kron([[1, 0] * nb_x, [0, 1] * nb_x] * nb_y, np.ones((s, s)))
	return C
\end{python}
	
\end{pcomment}


\begin{figure}[H]
\centering\caption{Sweden road sign to be used for corner detection.}%
 \includegraphics[width=5cm]{sweden_road.png}%
 \label{fig:harrs:enonce:road_sign}%
\end{figure}



\begin{qbox}
 \begin{itemize}
 \item Evaluate the extended maxima of the image.
 \item Only the strongest values of the cornerness measure should be kept. Two strategies can be employed in conjonction:
 \begin{itemize}\item Use a threshold value $t$ on $C$: the choice of this value is not trivial, and it strongly depends on the considered image. An adaptive method would be preferred.
 \item Keep only the $n$ strongest values.
 \end{itemize}
  \item The previous operations are affected by the borders of the image. Thus, eliminate the corner points near the borders.
  \item The detected corners may contain several pixels. Keep only the centroid of each cluster.
  
 \end{itemize}

\end{qbox}

\begin{pcomment}
\begin{phelp}
 There are numerous methods for computing local maxima. One can choose between \pinline{skimage.morphology.local_maxima}, \pinline{skimage.morphology.h_maxima}, \pinline{skimage.feature.peak_local_max}.
\end{phelp}

\end{pcomment}
