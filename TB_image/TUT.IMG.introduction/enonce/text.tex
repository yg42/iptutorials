\def\difficulty{1}
\sujet{Introduction to image processing}


\begin{note}
In this tutorial, you will discover the basic functions in order to load, manipulate and  display images. The main informations of the images will be retrieved, like size, number of channels, storage class, etc. Afterwards, you will be able to perform your first classic filters.

Do not forget that one of the main objectives is finding by yourself (in the documentation of the different modules) the usage of the appropriate functions. 
\end{note}

\noindent The different processes will be realized on the following images:

\vspace*{-8pt}
\begin{figure}[htbp]
\centering\caption{Image examples.}
\subfloat[Retinal vessels.]{\includegraphics[height=.28\linewidth]{retine.png}}
\hfill
\subfloat[Muscle cells.]{\includegraphics[height=.3\linewidth]{muscle.jpg}}
\hfill
\subfloat[Cornea cells (BIIGC, Univ. Jean Monnet, Saint-Etienne, France).]{\includegraphics[height=.3\linewidth]{cellules_cornee.jpg}}
\vspace*{-10pt}
\end{figure}

\vspace*{-15pt}

%%%%%%%%%%%%%%%%%%%%%%%%%%%%%%%%%%%%%%%%%%%%%%%%%%%%%%%%%%%%%%%%%%%%%%%%%%%%%%%%%%%%%%%%

\section{First manipulations}
\index{Image!Load}\index{Image!Display}

\begin{pcomment}
\begin{premark}
\begin{itemize}\item Image loading can be made by the use of the python function \pinline{skimage.io.imread}.
 \item 
The visualization of the image in the screen is realized by using the module \pinline{matplotlib.pyplot}.
\item Also import the numpy module with \pinline{import numpy as np}. All images with be manipulated as a numpy array.
\end{itemize}



\end{premark}
\end{pcomment}

\begin{mcomment}
\begin{mremark}Image loading can be made by the use of the \matlabregistered{} function \minline{imread}. 
The visualization of the image in the screen is realized either by the \matlabregistered{} function \minline{imagesc} or \minline{imshow}.
\end{mremark}
\end{mcomment}

\begin{qbox}
\begin{itemize}
\item Load and visualize the first image as below. Notice the differences.
 \item Look at the data structure of the image $I$ such as its size, type\dots.
\item Visualize the green component of the image.
Is it different from the red one? What is the most contrasted color component? Why? 
\item Enumerate some digital image file formats. 
What are their main differences? 
Try to write images with the JPEG file format with different compression ratios (0, 50 and 100), as while as the lossless compression, and compare.
\end{itemize}
\end{qbox}

\begin{mcomment}
 \begin{mremark}
  See \minline{imwrite}.
 \end{mremark}
\end{mcomment}

\begin{pcomment}
 \begin{premark}
  See \pinline{skimage.io.imsave}
 \end{premark}

\end{pcomment}
\index{Image!Save}

%%%%%%%%%%%%%%%%%%%%%%%%%%%%%%%%%%%%%%%%%%%%%%%%%%%%%%%%%%%%%%%%%%%%%%%%%%%%%%%%%%%%%%%%%%%%%%%%%%%
\section{Color quantization}
\index{Quantization}
Color quantization is a process that reduces the number of distinct colors used in an image, usually with the intention that the resulting image should be as visually similar as possible to the original image. In principle, a color image is usually quantized with 8 bits (i.e. 256 gray levels) for each color component.

\begin{qbox}
\begin{itemize}
\item By using the gray level image 'muscle', reduce the number of gray levels to  128, 64, 32, and visualize the different resulting images.
\item Compute the different image histograms and compare. 
\end{itemize}
\end{qbox}


\begin{pcomment}
 You can take advantage of the floor division in order to get integer value of the division, for example:
 \begin{sh}
>>> print(255//4)
63
 \end{sh}
 
\end{pcomment}



%%%%%%%%%%%%%%%%%%%%%%%%%%%%%%%%%%%%%%%%%%%%%%%%%%%%%%%%%%%%%%%%%%%%%%%%%%%%%%%%%%%%%%%%%%%%%%%%%%%
\vspace*{-8pt}
\section{Image histogram}
\index{Histogram}
An image histogram represents the gray level distribution in a digital image.
The histogram corresponds to the number of pixels for each gray level. It is equivalent to a probability density function.
\index{Probability density function}

\begin{mcomment}
You have to code a \matlabregistered{} function that computes the histogram of any gray level image, with the following prototype:
\begin{matlab}
function h = histogram(I)
\end{matlab}
\end{mcomment}

\begin{pcomment}


With \pinline{numpy}, you can use a the following code to plot the histogram:

\begin{python}
# I is the 2D grayscale image
h, edges = np.histogram(I, bins=256)
plt.plot(edges[:-1], h)
plt.show()
\end{python}

You have to code a python function with the following prototype:
\begin{python}
def histogram(I):
\end{python}

\end{pcomment}

\begin{qbox}
\begin{itemize}
 \item Test the numpy version of the histogram.
 \item Code your own version of this function and visualize the histogram of the image 'muscle.jpg'.
\end{itemize}

\end{qbox}


%%%%%%%%%%%%%%%%%%%%%%%%%%%%%%%%%%%%%%%%%%%%%%%%%%%%%%%%%%%%%%%%%%%%%%%%%%%%%%%%

\vspace*{-8pt}
\section{Linear mapping of the image in\-ten\-si\-ties}

The gray level range of the image 'cellules\_cornee.jpg' can be enhanced by a linear map\-ping such that the minimum (resp. maximum) gray level value of the resulting image is $0$ (resp. $255$). Mathematical\-ly, it consists in finding a function $f(x)=ax+b$ such that 
$f(min)=0$ and $f(max)=255$.

\begin{qbox}
\begin{itemize}
\item Load the image and find its extremal gray level values.
\item Adjust the intensities by a linear mapping into $[0, 255]$.

\item Visualize the resulting image and its histogram.
\end{itemize}
\end{qbox}


%%%%%%%%%%%%%%%%%
% filtering


\section{Low-pass filtering}
\index{Filtering!Low-pass}\index{Filtering!Convolution}

The different processes will be realized on the following images:
{
	\makeatletter
	\renewcommand\fs@ruled{\def\@fs@cfont{\bfseries}\let\@fs@capt\floatc@ruled
		\def\@fs@pre{\hrule height.8pt depth0pt \kern2pt}%
		\def\@fs@post{\kern2pt\hrule\relax}%
		\def\@fs@mid{\vskip2pt}%
		\let\@fs@iftopcapt\iftrue}
	\makeatother
\begin{figure}[H]
\centering
\subfloat[osteoblast cells]{\includegraphics[height=4.5cm]{osteoblaste.jpg}}
\hfill
\subfloat[blood cells]{\includegraphics[height=4.5cm]{blood.jpg}}
\vspace*{-10pt}
\end{figure}
}

Low-pass filtering aims to smooth the fast intensity variations of the image to be processed.
\begin{qbox}
Test the low-pass filters 'mean', 'median', 'min', 'max' and 'gaussian' on the noisy image 'blood cells'. 
\end{qbox}

	\begin{mcomment}
\begin{mremark}The \matlabregistered{} functions \minline{imfilter} and \minline{nlfilter} can be employed.
Be careful to the function options for border problems. Also, the \matlabregistered{} function \minline{fspecial} enables an operational window to be generated.
\end{mremark}
\end{mcomment}

\begin{pcomment}
\begin{premark}
The modules \cprotect{\href{https://scikit-image.org/docs/stable/api/skimage.filters.html}}{\pinline{scipy.ndimage.filters}} and  \cprotect{\href{https://scikit-image.org/docs/stable/api/skimage.filters.rank.html}}{\pinline{scipy.ndimage.filters.rank}} contain a lot of classical filters.
\end{premark}
\end{pcomment}
\begin{qbox}
Which filter is suitable for the restoration of this image?
\end{qbox}

%%%%%%%%%%%%%%%%%%%%%%%%%%%%%%%%%%%%%%%%%%%%%%%%%%%%%%%%%%%%%%%%%%%%%%%%%%%%%%%%%%%%%%%%%%%%%%%%%%%
\section{High-pass filtering}
\index{Filtering!High-pass}
\index{Laplacian}

High-pass filtering aims to smooth the low intensity variations of the image to be processed.
\begin{qbox}
\begin{itemize}
	\item Test the high-pass filters $HP$ on the two initial images in the following way: 
	$HP(f)=f-LP(f)$ where $LP$ is a low-pass filtering (see the previous exercise).
	\item Test the Laplacian \label{lbl:introduction:laplacien}(high-pass) filter on the two initial images.
	
\end{itemize}
\end{qbox}

\section{Convolution algorithm}

The general expression of the convolution, denoted by the operator $*$, is given by
$$g(x,y)= h *f(x,y)=\sum_{i=0}^{N-1}{\sum_{j=0}^{M-1}{ h(i, j)f(x-i,y-j)}}$$

with $f$ the original image, $g$ the filtered image and $h$ the convolution mask (a.k.a. kernel). 

\begin{pcomment}
The \cprotect{\href{https://docs.scipy.org/doc/scipy/reference/generated/scipy.signal.convolve2d.html}}{\pinline{scipy.signal.convolve2d}} is used for 2D convolution.
\end{pcomment}

\begin{qbox} Apply the convolution operation with the following convolution kernels, that define some classical filters:
 \begin{itemize}
 \item Mean filter
 $$\frac{1}{9}
\left[
\begin{array}{ccc}
1&1&1\\
1&1&1\\
1&1&1
\end{array}
\right]
$$
  \item Laplacian filter
$$
\left[
\begin{array}{ccc}
-1&-1&-1\\
-1&+8&-1\\
-1&-1&-1
\end{array}
\right]
$$

\item Gaussian filter:
$$\frac{1}{16}
\left[
\begin{array}{ccc}
1&2&1\\
2&4&2\\
1&2&1
\end{array}
\right]
$$
 \end{itemize}

\end{qbox}


\section{Derivative filters}
\index{Gradient!Prewitt}
\index{Gradient!Sobel}

Derivative filtering aims to detect the edges (contours) of the image to be processed.
\begin{qbox}
\begin{itemize}
	\item
Test the Prewitt and Sobel derivative filters (corresponding to first order derivatives) on the image 'blood cells' with the use of the following convolution masks:
$$
\left[
\begin{array}{ccc}
-1&0&+1\\
-1&0&+1\\
-1&0&+1
\end{array}
\right]
\left[
\begin{array}{rrr}
-1&-1&-1\\
0&0&0\\
+1&+1&+1
\end{array}
\right]
$$
$$
\left[
\begin{array}{ccc}
-1&0&+1\\
-2&0&+2\\
-1&0&+1
\end{array}
\right]
\left[
\begin{array}{rrr}
-1&-2&-1\\
0&0&0\\
+1&+2&+1
\end{array}
\right]
$$
\item Look at the results for the different gradient directions.
\item Define an operator taking into account the horizontal and vertical directions.
\end{itemize}
\end{qbox}

Remark : the edges could be also detected with the zero-crossings of the Laplacian filtering  (corresponding to second order derivatives).

 

\section{Enhancement filtering}
Enhancement filtering aims to enhance the contrast or accentuate some specific image characteristics.
\begin{qbox}
\begin{itemize}
	\item Test the enhancement filter $E$ on the image 'osteoblast cells' defined as:
	$E(f)=f+HP(f)$ where $HP$ is a Laplacian filter (see also  \iflabelexists{tutorial:image_enhancement}{tutorial \ref{tutorial:image_enhancement} for enhancement methods.}{tutorial about enhancement for classical methods.}).
	\item Parameterize the previous filter as:
	$E(f)=\alpha f+HP(f)$, where $\alpha\in\mathbb{R}$.
\end{itemize}
\end{qbox}


\section{Aliasing effect}
\index{Aliasing}

\begin{qbox}
	\begin{minipage}{0.6\textwidth}
\begin{itemize}
\item Create an image (as right) that contains rings as sinusoids. The function takes two input parameters: the sampling frequency and the signal frequency.
%\begin{center}
%\includegraphics[height=3.5cm]{moire.png}
%\includegraphics[height=4.25cm]{moire2.png}
%\end{center}
\item Look at the influence of the two varying frequencies. What do you observe? Explain the phenomenon from a theoretical point of view.
\end{itemize}
\end{minipage}
\hfill
\begin{minipage}{0.35\textwidth}
	\includegraphics[height=3.5cm]{moire.png}
\end{minipage}
\end{qbox}

\section{Open question}
Find an image filter for enhancing the gray level range of the image 'osteoblast cells'.

