\def\difficulty{1}
\sujet{Segmentation by region growing}
\begin{note}This tutorial proposes to program the region growing algorithm.\end{note}
\label{tutorial:regiongrowing}
\index{Segmentation!Region Growing}

\vspace*{-10pt}
\section{Introduction}
The region growing segmentation method starts from a seed. The initial region first contains this seed and then grows according to 
\begin{itemize}
 \item a growth mechanism (in this tutorial, the $N_8$ will be considered)
 \item an homogeneity rule (predicate function)
\end{itemize}

The algorithm is simple and barely only needs a predicate function:

\begin{algorithm}[H]
      \KwData{$I$: image}
      \KwData{seed: starting point}
      \KwData{queue: queue of points to considere}
      \KwResult{visited: boolean matrix, same size as $I$}
    \Begin{
      queue.enqueue( seed )\;
      
      \While{queue is not empty}{
	$p$ = queue.dequeue()\;
	
	\ForEach{neighbor of $p$}{
	  \If{not visited($p$) \textbf{ and} neighbor verifies predicate}{
	    queue.enqueue( neighbor )\;
	    visited( neighbor ) = true\;
	  }
	}
      }
      
      \Return visited
    }\caption{}
  \end{algorithm}%

  \begin{mcomment}
    
The difficulty of the code lays in the presence of a queue structure. For the purpose of simplicity, it is advised to use the java class \lstinline!java.util.LinkedList!, which is allowed within matlab:
\begin{matlab}
% create the queue structure by a Java object
queue = java.util.LinkedList;

% test it
p = [1;2];
queue.add(p);

r = queue.remove()

% test if queue is empty
if queue.isEmpty()
    % queue is empty
end
\end{matlab}

For picking up the coordinates of a pixel with the mouse in \matlabregistered{}, you can use the \minline{ginput} function:
\begin{matlab}
I = double(imread('cameraman.tif'));
[Sx, Sy] = size(I);
imshow(I,[]);

% seed
[x, y]=ginput(1);
seed = round([y;x]); % beware of inversion of coordinates
\end{matlab}
\end{mcomment}

\begin{pcomment}
To get the coordinate of the mouse click, use the matplotlib connect utilities. Define a function \pinline{def onpick}.
\begin{python}
# start by displaying a figure, 
# ask for mouse input (click)
fig = plt.figure ();
ax = fig.add_subplot (211);
ax.set_title ('Click on a point')
# load image
img = misc.ascent ();
ax.imshow (img, picker = True, cmap = plt.gray ());
# connect click on image to onpick function
fig.canvas.mpl_connect ('button_press_event', onpick);
plt.show ();
def onpick(event):
    """ connector """
\end{python}
\end{pcomment}


\section{Region growing implementation}
\begin{qbox}
The seed pixel is denoted $s$.
\begin{itemize}
 \item Code the predicate function: for an image $f$ and a pixel $p$, $p$ is in the same segment as $s$ implies $|f(s)-f(p)|\leq T$.
 \item Code a function that performs region growing, from a starting pixel (seed).
 \item Try others predicate functions like:
 \begin{itemize}
  \item pixel $p$ intensity is close to the region $\mathcal{R}$ mean value, i.e.: $$ |I(p)-m_\mathcal{R}|\leq T$$
  \item Threshold value $T$ varies depending on the region $\mathcal{R}$ and the intensity of the pixel $I(p)$. It can be chosen this way, with $\sigma$ and $m$ representing the standard deviation and the mean, respectively:
$$T=\left(1-\frac{\sigma_\mathcal{R}}{m_\mathcal{R}}\right)\cdot T_0$$
 \end{itemize}

\end{itemize}

\end{qbox}
